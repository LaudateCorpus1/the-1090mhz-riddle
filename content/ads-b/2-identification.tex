\chapter{Aircraft identification and category}

Within the group of ADS-B messages, the \emph{Aircraft Identification and Category} message is designed to broadcast the identification (also known as the \emph{callsign}), and the wake vortex category of the aircraft.

In this message, the Type Code can be from 1 to 4. The 56-bit ME filed consists of 10 parts and is structured as follows:

\begin{verbatim}
+------+------+------+------+------+------+------+------+------+------+
| TC,5 | CA,3 | C1,6 | C2,6 | C3,6 | C4,6 | C5,6 | C6,6 | C7,6 | C8,6 |
+------+------+------+------+------+------+------+------+------+------+

TC: Type code
CA: Aircraft category
C*: A character
\end{verbatim}

Here, number 6 represents the number of bits used to encode each of the characters.

\section{Identification (call sign)}
The aircraft identification included in the message is the callsign. Be aware that callsign is not a unique identifier of an aircraft, since different aircraft flying the same route at different times would share the same callsign.

The last 8 fields in the previous structure diagram represent the callsign characters. In order to decode each character, a lookup table is needed to map the corresponding decimal number (represented in binary code) to each character.

The character mapping is shown as follows:

\begin{verbatim}
#ABCDEFGHIJKLMNOPQRSTUVWXYZ##### ###############0123456789######
\end{verbatim}

Firstly, the \texttt{\#} symbols represent characters that are not used. In summary, characters and their decimal representations are as follows, where  \texttt{␣} symbol refers to a space character. 

\begin{verbatim}
A - Z :   1 - 26
0 - 9 :  48 - 57
    ␣ :  32
\end{verbatim}


If you are familiar with the ASCII (American Standard Code for Information Interchange) code, it is easy to identify that a callsign character is encoded using the lower 6 bits of the same character in ASCII.

\section{Wake vortex category}
The \texttt{CA} value in combination with \texttt{TC} value defines the wake vortex category of the aircraft. Note ADS-B has its own definition of wake categories, which is different from the common ICAO wake turbulence category definition. In Table \ref{tb:adsb_id_wake_category}, the definitions and related \texttt{TC} and \texttt{CA} codes are indicated.

\begin{table}[ht]
\caption{Wake vortex in ADS-B identification and category message}
\label{tb:adsb_id_wake_category}
\begin{tabular}{|l|l|l|}
\hline
\textbf{TC} & \textbf{CA} & \textbf{Category} \\ \hline\hline
1 & ANY & Reserved \\ \hline\hline
ANY & 0 & No category information \\ \hline\hline
2 & 1 & Surface emergency vehicle \\ \hline
2 & 3 & Surface service vehicle \\ \hline
2 & 4 - 7 & Ground obstruction \\ \hline\hline
3 & 1 & Glider, sailplane \\ \hline
3 & 2 & Lighter-than-air \\ \hline
3 & 3 & Parachutist, skydiver \\ \hline
3 & 4 & Ultralight, hang-glider, paraglider \\ \hline
3 & 5 & Reserved \\ \hline
3 & 6 & Unmanned aerial vehicle \\ \hline
3 & 7 & Space or transatmospheric vehicle \\ \hline\hline
4 & 1 & Light (less than 7000 kg) \\ \hline
4 & 2 & Medium 1 (between 7000 kg and 34000 kg) \\ \hline
4 & 3 & Medium 2 (between 34000 kg to 136000 kg) \\ \hline
4 & 4 & High vortex aircraft \\ \hline
4 & 5 & Heavy (larger than 136000 kg) \\ \hline
4 & 6 & High performance (\textgreater 5 g acceleration) and high speed (\textgreater 400 kt) \\ \hline
4 & 7 & Rotorcraft \\ \hline
\end{tabular}
\end{table}


\section{Decoding example}

Let us use the following raw message as an example for decoding:

\begin{verbatim}
8D4840D6202CC371C32CE0576098
\end{verbatim}

The ME field is:

\begin{verbatim}
HEX: 202CC371C32CE0
BIN: 00100000001011001100001101110001110000110010110011100000
\end{verbatim}

The structure of the ME field can be decomposed as follows:

\begin{verbatim}
+-----+---+------+------+------+------+------+------+------+------+
|00100|000|001011|001100|001101|110001|110000|110010|110011|100000|
+-----+---+------+------+------+------+------+------+------+------+
|TC   |CA |11    |12    |13    |49    |48    |50    |51    |32    |
+-----+---+------+------+------+------+------+------+------+------+
|4    |0  |K     |L     |M     |1     |0     |2     |3     |_     |
+-----+---+------+------+------+------+------+------+------+------+
\end{verbatim}

With \texttt{TC=4}, we can confirm it is an identification and category message. The decoded identity (callsign) of the aircraft is \texttt{KLM1023} (with the trailing space ignored). With \texttt{CA=0}, we can see that the aircraft did not transmit any information on the wake vortex category.

\begin{notebox}{Try it out}
Using \texttt{pyModeS}, we can decode the callsign as follows:

\begin{verbatim}
import pyModeS as pms

category = pms.adsb.category("8D4840D6202CC371C32CE0576098")
callsign = pms.adsb.callsign("8D4840D6202CC371C32CE0576098")
\end{verbatim}

\end{notebox}
 
  