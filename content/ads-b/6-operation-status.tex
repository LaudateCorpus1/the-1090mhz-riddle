\chapter{Aircraft operation status}\label{aircraft-operation-status}

The aircraft operational status message is designed to provide various information on an aircraft. It is transmitted with Type Code 31 (\texttt{TC=31}). The structures of this message type differ significantly over different ADS-B versions. The message has been defined in all ADS-B versions. But in practice, it is not implemented in ADS-B version 0. From version 1 onward, the operational status includes more information, such as ADS-B version, accuracy, and integrity indicators. 


\section{Version 0}

In version 0, the structure of the message is shown in Table \ref{tb:adsb-operational-status-v0}. Two main blocks contain the status information, which includes operational capabilities (16 bits) and operational status (16 bits). Each block contains four parameters, and each parameter is encoded with 4 bits.

\begin{table}[ht]
\caption{Aircraft operational status (Version 0)}
\label{tb:adsb-operational-status-v0}
\footnotesize
\begin{tabular}{|l|c|c|c|c|}
\hline
\textbf{FIELD} &  & \textbf{MSG} & \textbf{ME} & \textbf{BITS} \\ \hline
Type code = 31 (Binary: 11111) & TC & 33--37 & 1--5 & 5 \\ \hline
Sub-type code = 0 (Binary: 000) & ST & 38--40 & 6--8 & 3 \\ \hline
Enroute operational capabilities & CC4 & 41--44 & 9--12 & 4\\
Terminal area operational capabilities & CC3 & 45--48 & 13--16 & 4\\
Approach/landing operational capabilities & CC2 & 49--52 & 17--21 & 4\\
Surface operational capabilities & CC1 & 53--56 & 22--24 & 4 \\ \hline
Enroute operational status & OM4 & 57--60 & 25--28 & 4\\
Terminal area operational status & OM3 & 61--64 & 29--32 & 4\\
Approach/landing operational status & OM2 & 65--68 & 33--36 & 4 \\
Surface operational status  & OM1 & 69--72 & 36--39 & 4\\ \hline
Reserved & & 73--88 & 41--56 & 16 \\ \hline
\end{tabular}
\end{table}

However, even though all the parameter fields are defined, all 
\texttt{CC} and \text{OM} bits are reserved in the third and fourth blocks according to \cite{icao9871v1}, which eventually made the definitions unusable for version 0 transponders. As version 0 transponders do not transmit any operation status messages, the absence of TC=31 messages can help identify transponder version 0, which is discussed in Chapter \ref{chap:adsb-basic}.

\section{Version 1}

From version 1 onward, the operational status report is implemented and broadcast by the transponder. Table \ref{tb:adsb-operational-status-v1} lists the structure of operational message in version 1.

\begin{table}[ht]
\caption{Aircraft operational status (Version 1)}
\label{tb:adsb-operational-status-v1}
\footnotesize
\begin{tabular}{|l|c|c|c|c|}
\hline
\textbf{FIELD} &  & \textbf{MSG} & \textbf{ME} & \textbf{BITS} \\ \hline \hline
\textbf{Type code = 31} (Binary: 11111) & TC & 33--37 & 1--5 & 5 \\ \hline
\textbf{Sub-type code} & ST & 38--40 & 6--8 & 3\\
0: airborne, 1: surface, 2--7: reserved &&&&  \\ \hline
\textbf{Capacity class codes} & CC & 41--56 & 9--24 & 16 \\
ST=0: Airborne capacity class codes (16 bits) &&&& \\
ST=1: Surface capacity codes (12 bits) &&&& \\
\quad\qquad+ Length/width code (4 bits) &&&& \\ \hline
\textbf{Operational mode codes} & OM & 57--72 & 25--40 & 16 \\ \hline
\textbf{ADS-B version number} & Ver & 73--75 & 41--43 & 3\\ 
0: Comply with DOC 9871, Appendix A &&&& \\
1: Comply with DOC 9871, Appendix B &&&& \\
2--7: Reserved &&&& \\ \hline
\textbf{NIC supplement} & NICs & 76 & 44 & 1 \\ \hline
\textbf{Navigational accuracy category - position} & NACp & 77--80 & 45--48 & 4 \\ \hline
\textbf{ST=0: Barometric altitude quality} & BAQ & 81--82 & 49--50 & 2\\
\textbf{ST=1: Reserved} &&&& \\ \hline
\textbf{Surveillance integrity level} & SIL & 83--84 & 51--52 & 2 \\ \hline
\textbf{ST=0: Barometric altitude integrity} & & 85 & 53 & 1\\
\textbf{ST=1: Track angle or heading} & & & &  \\ \hline
\textbf{Horizontal reference direction} & HRD & 86 & 54 & 1 \\ \hline
\textbf{Reserved} & & 87--88 & 55--56 & 2 \\ \hline
\end{tabular}
\end{table}

Bits 73 to 86, which are unused in the previous version, are now defined with new parameters. These parameters include ADS-B version number and various indicators related to the accuracy of the state measurements broadcast by the aircraft in other types of ADS-B messages.

It is worth noting that the definition of sub-types for some fields allow us to differentiate between airborne status messages and surface status messages.


\section{Version 2}

Compared to the previous update, the changes in the operational status report from version 1 to 2 are minimal. Several fields are renamed and a SIL supplement bit is added to bit-87. The structure is shown in Table \ref{tb:adsb-operational-status-v2}.


\begin{table}[ht]
\caption{Aircraft operational status (Version 2)}
\label{tb:adsb-operational-status-v2}
\footnotesize
\begin{tabular}{|l|c|c|c|c|}
\hline
\textbf{FIELD} &  & \textbf{MSG} & \textbf{ME} & \textbf{BITS} \\ \hline
\textbf{Type code = 31} (Binary: 11111) & TC & 33--37 & 1--5 & 5 \\ \hline
\textbf{Sub-type code} & ST & 38--40 & 6--8 & 3 \\
0: Airborne status message &&&& \\
1: Surface status message &&&& \\ \hline
\textbf{Capacity class codes} & CC & 41--56 & 9--24 & 16 \\
ST=0: Airborne capacity class codes (16 bits) &&&& \\
ST=1: Surface capacity codes (12 bits) &&&& \\
\quad\qquad+ Length/width code (4 bits) &&&& \\ \hline
\textbf{Operational mode codes} & OM & 57--72 & 25--40 & 16 \\
\textbf{ST=0: Airborne operational mode codes} &&&&\\
\textbf{ST=1: Surface operational mode codes} &&&& \\ \hline
\textbf{ADS-B version number} & Ver & 73--75 & 41--43 & 3\\ 
0: Comply with DOC 9871, Appendix A &&&&\\ 
1: Comply with DOC 9871, Appendix B &&&&\\ 
2: Comply with DOC 9871, Appendix C &&&&\\ 
3--7: Reserved &&&& \\ \hline
\textbf{NIC supplement - A} & NICa & 76 & 44 &  \\ \hline
\textbf{Navigational accuracy category - position} & NACp & 77--80 & 45--48 & 4 \\ \hline
\textbf{ST=0: Geometric vertical accuracy} & GVA & 81--82 & 49--50 & 2\\
\textbf{ST=1: Reserved} &&&& \\ \hline
\textbf{Source integrity level} & SIL & 83--84 & 51--52 & 2 \\ \hline
\textbf{ST=0: Barometric altitude integrity} & & 85 & 53 & 1\\
\textbf{ST=1: Track angle or heading} &&&& \\ \hline
\textbf{Horizontal reference direction} & HRD & 86 & 54 & 1 \\ \hline
\textbf{SIL supplement} & SILs & 87 & 55 & 1 \\ \hline
\textbf{Reserved} &  & 88 & 56 & 1 \\ \hline
\end{tabular}
\end{table}
