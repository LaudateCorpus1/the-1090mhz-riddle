\chapter{All-call reply}

In chapter \ref{chap:intro}, Figure \ref{fig:mode_s_inter_mode} shows that Mode S all-call replies can be generated by Mode S transponder to answer Mode S only all-call interrogations and Mode A/C/S all-call interrogations.

Downlink format 11 is used for the all-call reply, and the length of the messages is 56 bits. The structure of an all-call reply message is quite simple. It only contains four fields, which are shown in Table \ref{tb:df11_structure}.

\begin{table}[ht]
\caption{All-call reply}
\label{tb:df11_structure}
\begin{tabular}{|l|l|l|l|}
\hline
\textbf{FIELD} & \textbf{} & \textbf{MSG} & \textbf{BITS} \\ \hline
Downlink format & DF & 1-5 & 5 \\ \hline
Capability & CA & 6-8 & 3 \\ \hline
Address announced & AA & 9-32 & 24 \\ \hline
Parity/interrogator identifier & PI & 33-56 & 24 \\ \hline
\end{tabular}
\end{table}

The fields can be decoded as follows:

\begin{itemize}
  \item The definition of transponder capability is the same as in ADS-B messages (Table \ref{tb:transponder_capability} in chapter \ref{chap:adsb-basic}).

  \item The address refers to the 24-bit transponder address.

  \item The decoding of PI is similar to the decoding of ADS-B parity (chapter \ref{chap:adsb_parity}). For a message with downlink format 11, PI is overlaid with the interrogator identifier.\footnote{In fact, PI in ADS-B is also overlaid with the interrogator identifier. However, since ADS-B is not interrogation based, the identifier is set to 0. Thus, PI is always the same as the CRC remainder in ADS-B.} Hence, assuming there is no error in the data, the CRC will produce the interrogator identifier code for all-call replies.

\end{itemize}


A decoding example is shown as follows:

\begin{verbatim}
MSG HEX:  5D484FDEA248F5
MSG BIN:  01011   101    010010000100111111011110  101000100100100011110101
          [DF=11] [CA=5] [AA=484FDE]               [CRC=22]
\end{verbatim}

From the capability (CA=5), we can see the aircraft has a Level 2+ transponder, with the ability to set CA to 7 and that it is airborne. The CRC remainder indicates the interrogator identifier code is 22.

\begin{notebox}{Try it out}
Using \texttt{pyModeS}, we can obtain the ICAO address as: 

\begin{verbatim}
import pyModeS as pms

msg = "5D484FDEA248F5"
icao = pms.icao(msg)
\end{verbatim}

\end{notebox}