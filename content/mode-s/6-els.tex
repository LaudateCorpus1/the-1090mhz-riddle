\chapter{Mode~S elementary surveillance}

Mode~S Elementary Surveillance (ELS) provides a set of basic functionalities that are supported by the Mode~S transponders. These basic functionalities include the reporting  of aircraft identity, altitude, transponder capability, and flight status. The set of BDS codes included in ELS are \texttt{1,0}, \texttt{1,7}, \texttt{2,0}, and \texttt{3,0}.

\section{Data link capability report (BDS 1,0)}

This message is designed to report the data link capability of the installed Mode~S transponder. Table \ref{tb:bds10} shows all fields in this message.

\begin{table}[ht]
\centering
\caption{Data link capability report (BDS 1,0), MB field}
\label{tb:bds10}
\begin{tabular}{|l|l|l|l|}
\hline
\textbf{FIELD} & \textbf{MSG} & \textbf{MB} & \textbf{BITS} \\ \hline
BDS Code {[}0001 0000{]} & 33--40 & 1--8 & 8 \\ \hline
Configuration flag & 41 & 9 & 1 \\ \hline
Reserved {[}00000{]} & 42--46 & 10--14 & 6 \\ \hline
Overlay Command Capability (OCC) & 47 & 15 & 1 \\ \hline
Reserved for ACAS & 48 & 16 & 1 \\ \hline
Mode~S subnetwork version number & 49--55 & 17--23 & 7 \\ \hline
Transponder enhanced protocol indicator & 56 & 24 & 1 \\ \hline
Mode~S specific services capability & 57 & 25 & 1 \\ \hline
Uplink ELM average throughput capacity & 58--60 & 26--28 & 3 \\ \hline
Downlink ELM throughput & 61--64 & 29--32 & 4 \\ \hline
Aircraft identification capability & 65 & 33 & 1 \\ \hline
Squitter capability subfield (SCS) & 66 & 34 & 1 \\ \hline
Surveillance identifier code (SIC) & 67 & 35 & 1 \\ \hline
Common usage GICB capability report & 68 & 36 & 1 \\ \hline
Reserved for ACAS & 69--72 & 37--40 & 4 \\ \hline
Data terminal equipment (DTE) status & 73--80 & 41--56 & 16 \\ \hline
\end{tabular}
\end{table}


In the data link capability report, the first eight bits indicate the BDS number, which is \texttt{1,0}, or \texttt{0001 0000} in binary format. 

The definitions of the other fields are explained as follows:

1) ACAS related bits can be decoded as:

\begin{verbatim}
MSG     MB      Coding
--------------------------------------------------------------
48      16      0: ACAS failed or on standby
                1: ACAS operating
--------------------------------------------------------------
69      37      0: Hybrid surveillance not operational
                1: Hybrid surveillance fitted and operational
--------------------------------------------------------------
70      38      0: ACAS generating TAs only
                1: ACAS generating TAs and RAs
--------------------------------------------------------------
72|71   40|39   0|0: RTCA/DO-185 (pre-ACAS)
                0|1: RTCA/DO-185A
                1|0: RTCA/DO-185B or EUROCAE ED 143
                1|1: Reserved for future versions
\end{verbatim}

2) Valid values for \emph{Mode~S subnetwork version number} are from 0 to 5, corresponds to the compliance to a different version of ICAO documentations. Numbers 6 to 127 are currently unassigned. Numbers 0 to 5 can be interpreted as:

\begin{verbatim}
0:  Subnetwork not available
1:  ICAO Doc 9688 (1996)
2:  ICAO Doc 9688 (1998)
3:  ICAO Annex 10, Vol III, Amdt 77
4:  ICAO Doc 9871 (Ed 1), RTCA DO-181D, EUROCAE ED-73C
5:  ICAO Doc 9871 (Ed 2), RTCA DO-181E, EUROCAE ED-73E
>5: Reserved for future use
\end{verbatim}

3) \emph{Overlay Command Capability} indicates whether the transponder supports BDS overlay (Data Parity).

4) When \emph{Transponder enhanced protocol indicator} bit is set to \texttt{1}, the transponder is a Level 5 transponder. Value \texttt{0} indicates a Level 2 to 4 transponder.

5) When \emph{Mode~S specific services capability} bit is set to \texttt{1}, at least one Mode~S specific service is supported.\footnote{The additional specific services are BDS codes other than \texttt{0,2}, \texttt{0,3}, \texttt{1,0}, \texttt{1,7}-\texttt{1,C}, \texttt{2,0}, and \texttt{3,0}}.

6) \emph{Aircraft identification capability} indicates availability of identification (Callsign).

7) When \emph{Squitter capability subfield} bit is set to \texttt{1}, both BDS 0,5 and \texttt{0,6} registers have been updated in the past 9 to 11 seconds.

8) \emph{Surveillance identifier code} determines whether the transponder has the surveillance identification code capability.

9) \emph{Common usage GICB capability report} is set to \texttt{1} every time the GICB capacity report (BDS 1,7) is changed. Register \texttt{1,7} is sampled every minute to check for changes.

\section{Common usage GICB capability report (BDS 1,7)}

This message is designed to report Common Usage Ground-initiated Comm-B (GICB) capabilities. The fields in this report are shown in Table \ref{tb:bds17}. In this table, each bit indicates whether the corresponding service is available from the transponder. 

A bit is set to \texttt{1} when the corresponding register has a valid input that has been updated at the required rate. This means that the same aircraft would respond with different GICB reports due to the availability of the relevant data.

\begin{table}[ht]
\footnotesize
\centering
\caption{Common usage GICB capability report (BDS 1,0), MB field}
\label{tb:bds17}
\begin{tabular}{|l|l|l|l|}
\hline
\textbf{FIELD} & \textbf{MSG} & \textbf{MB} & \textbf{BITS} \\ \hline
0,5 Extended squitter airborne position & 33 & 1 & 1 \\ \hline
0,6 Extended squitter surface position & 34 & 2 & 1 \\ \hline
0,7 Extended squitter status & 35 & 3 & 1 \\ \hline
0,8 Extended squitter identification and category & 36 & 4 & 1 \\ \hline
0,9 Extended squitter airborne velocity information & 37 & 5 & 1 \\ \hline
0,A Extended squitter event-driven information & 38 & 6 & 1 \\ \hline
2,0 Aircraft identification & 39 & 7 & 1 \\ \hline
2,1 Aircraft registration number & 40 & 8 & 1 \\ \hline
4,0 Selected vertical intention & 41 & 9 & 1 \\ \hline
4,1 Next waypoint identifier & 42 & 10 & 1 \\ \hline
4,2 Next waypoint position & 43 & 11 & 1 \\ \hline
4,3 Next waypoint information & 44 & 12 & 1 \\ \hline
4,4 Meteorological routine report & 45 & 13 & 1 \\ \hline
4,5 Meteorological hazard report & 46 & 14 & 1 \\ \hline
4.8 VHF channel report & 47 & 15 & 1 \\ \hline
5,0 Track and turn report & 48 & 16 & 1 \\ \hline
5,1 Position coarse & 49 & 17 & 1 \\ \hline
5,2 Position fine & 50 & 18 & 1 \\ \hline
5,3 Air-referenced state vector & 51 & 19 & 1 \\ \hline
5,4 Waypoint 1 & 52 & 20 & 1 \\ \hline
5,5 Waypoint 2 & 53 & 21 & 1 \\ \hline
5,6 Waypoint 3 & 54 & 22 & 1 \\ \hline
5,F Quasi-static parameter monitoring & 55 & 23 & 1 \\ \hline
6,0 Heading and speed report & 56 & 24 & 1 \\ \hline
Reserved for aircraft capability & 57 & 25 & 1 \\ \hline
Reserved for aircraft capability & 58 & 26 & 1 \\ \hline
E,1 Reserved for Mode~S BITE (Built In Test Equipment) & 59 & 27 & 1 \\ \hline
E,2 Reserved for Mode~S BITE (Built In Test Equipment) & 60 & 28 & 1 \\ \hline
F,1 Military applications & 61 & 29 & 1 \\ \hline
Reserved & 62--80 & 30--56 & 17 \\ \hline
\end{tabular}
\end{table}

We can use the following message as an example:

\begin{verbatim}
MSG:      A0000638FA81C10000000081A92F
MB:               FA81C100000000
MB BIN:   11111010100000011100000100000000000000000000000000000000
\end{verbatim}

From the MB field, we can identify that following bits are set to \texttt{1}:

\begin{verbatim}
Bit: 1, 2, 3, 4, 5, 7, 9, 16, 17, 18, 24
\end{verbatim}

Hence, BDS codes supported by the transponder are: 
\begin{verbatim}
BDS05 BDS06 BDS07 BDS08 BDS09     < ADSB
BDS20                             < ELS
BDS40 BDS50 BDS51 and BDS60       < EHS
\end{verbatim}
  
\begin{notebox}{Try it out}
Using pyModeS, we can decode the GICB information as: 

\begin{verbatim}
import pyModeS as pms

msg = "A0000638FA81C10000000081A92F"
capabilities = pms.commb.cap17(msg)
\end{verbatim}

A list of supporting BDS code will be returned:

\begin{verbatim}
BDS05, BDS06, BDS07, BDS08, BDS09, BDS20, 
BDS40, BDS50, BDS51, BDS52, BDS60
\end{verbatim}

\end{notebox}

\clearpage

\section{Aircraft identification (BDS 2,0)}

Similar to an ADS-B aircraft identification message, the callsign of an aircraft can be decoded from BDS 2,0 messages. The structure of the MB field is defined in Table \ref{tb:bds30}.

\begin{table}[ht]
\centering
\caption{Aircraft identification (BDS 2,0), MB field}
\label{tb:bds30}
\begin{tabular}{|l|l|l|l|}
\hline
\textbf{FIELD} & \textbf{MSG} & \textbf{MB} & \textbf{BITS} \\ \hline
BDS Code {[}0010 0000{]} & 33--40 & 1--8 & 8 \\ \hline
Character 1 & 41--46 & 9--14 & 6 \\ \hline
Character 2 & 47--52 & 15--20 & 6 \\ \hline
Character 3 & 53--58 & 21--26 & 6 \\ \hline
Character 4 & 59--64 & 27--32 & 6 \\ \hline
Character 5 & 65--70 & 33--38 & 6 \\ \hline
Character 6 & 71--76 & 39--44 & 6 \\ \hline
Character 7 & 77--82 & 45--50 & 6 \\ \hline
Character 8 & 83--88 & 51--56 & 6 \\ \hline
\end{tabular}
\end{table}


The first eight bits indicate the BDS code \texttt{0010 0000} (\texttt{2,0} in hexadecimal). Each of the callsign characters is represented by six bits. We first need to convert the binary numbers to decimals. Each decimal value corresponds to the index of the letter in the following character map:

\begin{verbatim}
#ABCDEFGHIJKLMNOPQRSTUVWXYZ##### ###############0123456789######
\end{verbatim}

It is worth noting that this character mapping is the same as the one used for ADS-B identification, which also corresponds to a part of the ASCII code map.

The following is an example on how to decode the callsign from a sample BDS 2,0 message:

\begin{verbatim}
MSG:  A000083E202CC371C31DE0AA1CCF
MB:           202CC371C31DE0
-----------------------------------------------------------------------------
MB BIN:  0010 0000 001011 001100 001101 110001 110000 110001 110111 100000
HEX:        2    0
DEC:                 11     12     13     49     48     49     55     32
CHR:                 K      L      M      1      0      1      7     [SPACE]
-----------------------------------------------------------------------------
ID:   KLM1017
\end{verbatim}

\begin{notebox}{Try it out}
Using pyModeS, we can decode the callsign in this message as: 

\begin{verbatim}
import pyModeS as pms

msg = "A000083E202CC371C31DE0AA1CCF"
callsign = pms.commb.cs20(msg)
\end{verbatim}

Callsign \texttt{KLM1017\_} will be returned by the previous function. The space character is replace by \texttt{\_} in pyModeS.
\end{notebox}


\clearpage

\section{ACAS active resolution advisory (BDS 3,0)}

The BDS 3,0 message is used to report resolution advisories (RA) generated by ACAS equipment. The structure of the MB field is defined in Table \ref{tb:bds20}.\footnote{The decoding is also subject to the version of aircraft's TCAS version. For example, TCAS Version 7.0 does not comply with this interpretations.}

\begin{table}[ht]
\centering
\caption{ACAS active resolution advisory (BDS 3,0), MB field}
\label{tb:bds20}
\begin{tabular}{|l|l|l|l|}
\hline
\textbf{FIELD} & \textbf{MSG} & \textbf{MB} & \textbf{BITS} \\ \hline
BDS Code {[}0011 0000{]} & 33--40 & 1--8 & 8 \\ \hline
Active resolution advisories & 41--54 & 9--22 & 14 \\ \hline
Resolution advisory complements record & 55--58 & 23--26 & 4 \\ \hline
RA terminated & 59 & 27 & 1 \\ \hline
Multiple threat encounter & 60 & 28 & 1 \\ \hline
Threat type indicator & 61--62 & 29--30 & 2 \\ \hline
Threat identity data & 63--88 & 31--56 & 26 \\ \hline
\end{tabular}
\end{table}


1) The first 8 bits show the BDS code of \texttt{3,0} or \texttt{0011 0000} in binary format.

2) The 14-bit active resolution advisories (ARA) indicate the characteristics of the RA generated by the ACAS associated with the transponder. To decode the information, we need to decode MB bit 9 (MSG bit 41) and MB bit 28 (MSG bit 60). Table \ref{tb:bds30_mb9_mb28} shows how to interpret these two bits.

\begin{table}[ht]
\centering
\caption{BDS 3,0 MB bits 9 and 28}
\label{tb:bds30_mb9_mb28}
\begin{tabular}{|l|l|p{7cm}|}
\hline
\textbf{MB:9} & \textbf{MB:28} & \textbf{} \\ \hline
0 & 0 & No RA has been generated \\ \hline
0 & 1 & Multiple threats, RA is intended to provide vertical separation below some threats and above some other threats \\ \hline
1 & 0 & Only one threat \\ \hline
0 & 1 & Multiple threats, RA is intended to provide vertical separation in the same direction \\ \hline
\end{tabular}
\end{table}

MB bits 16--22 are reserved for ACAS III. When \texttt{MB:9=1}, the meanings of MB bits from 10 to 15 are shown in Table \ref{tb:bds30_mb10-22_1}.

\begin{table}[]
\centering
\caption{BDS 3,0 MB bits 10 to 15 (MB:9=1)}
\label{tb:bds30_mb10-22_1}
\begin{tabular}{|l|l|l|}
\hline
\textbf{MB} & \textbf{Value} & \textbf{} \\ \hline
10 & 0 & RA is preventive \\ \hline
 & 1 & RA is corrective \\ \hline
11 & 0 & Upward sense RA has been generated \\ \hline
 & 1 & Downward sense RA has been generated \\ \hline
12 & 0 & RA is not increased rate \\ \hline
 & 1 & RA is increased rate \\ \hline
13 & 0 & RA is not a sense reversal \\ \hline
 & 1 & RA is a sense reversal \\ \hline
14 & 0 & RA is not altitude crossing \\ \hline
 & 1 & RA is altitude crossing \\ \hline
15 & 0 & RA is vertical speed limit \\ \hline
 & 1 & RA is positive \\ \hline
\end{tabular}
\end{table}

When \texttt{MB:9=0} and \texttt{MB:28=1}, the meanings of MB bits from 10 to 15 are shown in Table \ref{tb:bds30_mb10-22_2}.

\begin{table}[]
\centering
\caption{BDS 3,0 MB bits 10 to 15 (MB:9=0 MB:28=1)}
\label{tb:bds30_mb10-22_2}
\begin{tabular}{|l|l|l|}
\hline
\textbf{MB} & \textbf{Value} & \textbf{} \\ \hline
10 & 0 & RA does not require a correction in the upward sense \\ \hline
 & 1 & RA requires a correction in the upward sense \\ \hline
11 & 0 & RA does not require a positive climb \\ \hline
 & 1 & RA requires a positive climb \\ \hline
12 & 0 & RA does not require a correction in the downward sense \\ \hline
 & 1 & RA requires a correction in the downward sense \\ \hline
13 & 0 & RA does not require a positive descend \\ \hline
 & 1 & RA requires a positive descend \\ \hline
14 & 0 & RA does not require a crossing \\ \hline
 & 1 & RA requires a crossing \\ \hline
15 & 0 & RA is not a sense reversal \\ \hline
 & 1 & RA is a sense reversal \\ \hline
\end{tabular}
\end{table}

\newpage 
2) The resolution advisory complements (RAC) record consists of four bits, and each bit has the following meaning:

\begin{verbatim}
MB:23=1   Do not pass below
MB:24=1   Do not pass above
MB:25=1   Do not turn left
MB:26=1   Do not turn right
\end{verbatim}


3) An RA terminated bit indicates whether previously generated RA has been terminated.


4) The threat type indicator contains two bits with the following meaning:

\begin{verbatim}
00    No identity data in threat identity data
01    Threat identity data contains a Mode~S transponder address
10    Threat identity data contains altitude, range, and bearing
11    Not assigned
\end{verbatim}

When the threat type indicator is \texttt{01}, MB bits 31--54 contain the 24-bit Mode~S transponder address and the last two bits are set to zero.

When the threat type indicator is \texttt{10}, MB bits 31--56 is divided into three different segments.

\begin{itemize}
  \item MB bits 31--43 contain the 13-bit threat altitude code. It is the same structure as the altitude code from section \ref{sec:alt_code}.
  \item MB bits 44--50 contain the most recent threat range from ACAS. When it is 0, no range estimation is available. When it is 127, the range is greater than 12.55 nautical miles. For other values, the range is calculated as $(n-1) / 10 \pm 0.05$ nautical miles.
  \item MB bits 51--56 contain the most recent estimated bearing of the threat aircraft, relative to their own heading. When it is 0, no estimation is available. Values larger than 60 are not assigned. For other values (from 1 to 60), the bearing range is calculated as $[6(n-1), 6n]$ in degrees.
\end{itemize}
