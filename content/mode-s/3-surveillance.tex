\chapter{Surveillance replies} \label{chap:surv_reply}

Surveillance replies consist of short messages (56 bits) that respond to the selective interrogations of the secondary surveillance radar based on the 24-bit transponder addresses of the aircraft. Two types of information are transmitted: altitude (DF=4) and identity (DF=5).


\section{Message structure}

The structure of altitude and identity surveillance replies are similar, as shown in Tables \ref{tb:df_4_structure} and \ref{tb:df_5_structure}. The main difference is that bits 20 to 32 are used to encode either the altitude or the squawk code.

\begin{table}[ht]
  \centering
  \caption{Surveillance altitude reply (DF=4)}
  \label{tb:df_4_structure}
  \begin{tabular}[t]{|l|l|l|l|}
  \hline
  \textbf{FIELD} & \textbf{} & \textbf{MSG} & \textbf{BITS} \\ \hline
  Downlink format   & DF & 1-5    & 5   \\ \hline
  Flight status     & FS & 6-8    & 3   \\ \hline
  Downlink request  & DR & 9-13   & 5   \\ \hline
  Utility message   & UM & 14-19  & 6   \\ \hline
  Altitude code     & AC & 20-32  & 13  \\ \hline
  Address parity    & AP & 33-56  & 24  \\ \hline
  \end{tabular}
\end{table}

\begin{table}[ht]
  \centering
  \caption{Surveillance identity reply (DF=5)}
  \label{tb:df_5_structure}
  \begin{tabular}[t]{|l|l|l|l|}
  \hline
  \textbf{FIELD} & \textbf{} & \textbf{MSG} & \textbf{BITS} \\ \hline
  Downlink format   & DF & 1-5    & 5   \\ \hline
  Flight status     & FS & 6-8    & 3   \\ \hline
  Downlink request  & DR & 9-13   & 5   \\ \hline
  Utility message   & UM & 14-19  & 6   \\ \hline
  Identity code     & ID & 20-32  & 13  \\ \hline
  Address parity    & AP & 33-56  & 24  \\ \hline
  \end{tabular}
\end{table}
  

The definitions of the common fields are:

\begin{itemize}
  \item \emph{Flight status (FS)}: 3 bits, shows status of alert, special position pulse (SPI, in Mode~A only) and aircraft status (airborne or on-ground). The field is interpreted as:

  \begin{quote}
    \small
    \texttt{000}: no alert, no SPI, aircraft is airborne \\
    \texttt{001}: no alert, no SPI, aircraft is on-ground \\
    \texttt{010}: alert, no SPI, aircraft is airborne \\
    \texttt{011}: alert, no SPI, aircraft is on-ground \\
    \texttt{100}: alert, SPI, aircraft is airborne or on-ground \\
    \texttt{101}: no alert, SPI, aircraft is airborne or on-ground \\
    \texttt{110}: reserved \\
    \texttt{111}: not assigned
  \end{quote}

  \item \emph{Downlink request (DR)}: 5 bits, contains the type of request. In surveillance replies, only values 0, 1, 4, and 5 are used. The field can be decoded as:

  \begin{quote}
    \small
    \texttt{00000}: no downlink request \\
    \texttt{00001}: request to send Comm-B message \\
    \texttt{00100}: Comm-B broadcast messages 1 available \\
    \texttt{00101}: Comm-B broadcast messages 2 available
  \end{quote}

  \item \emph{Utility message (UM)}: 6 bits, contains transponder communication status information. 
  
  \begin{itemize}
    \item IIS: The first 4 bits of UM indicate the interrogator identifier code. 
    \item IDS: The last 2 bits indicate the type of reservation made by the interrogator.

    \begin{quote}
      \small
      \texttt{00}: no information \\
      \texttt{01}: IIS contains Comm-B interrogator identifier code \\
      \texttt{10}: IIS contains Comm-C interrogator identifier code \\
      \texttt{11}: IIS contains Comm-D interrogator identifier code \\
    \end{quote}
  
  \end{itemize}

  


\end{itemize}



\section{Altitude code} \label{sec:alt_code}

The 13-bit altitude code can be encoded in 25-ft increments, 100-ft increments, or in metric units (m). The 7th bit (MSG bit 28) is defined as the M bit. When the M bit is set to \0, the 9th bit (MSG bit 30) is defined as the Q bit.

The decoding rules are as follows:

\begin{itemize}
  \item When all bits are set to \0, altitude information is not available or invalid.

  \item When M=1, removing the M bit, the remaining 12 bits represent the altitude in meters.

  \item When M=0 and Q=1, removing the M and Q bits, the remaining 11 bits encode the altitude with 25 ft increments. Denote $N$ as the decimal value of the remaining 11 bits. The altitude is calculated as $25 \times N - 1000$ ft.

  \item When M=0 and Q=0, the remaining bits are defined as follows:

\begin{verbatim}
                                M        Q
+----+----+----+----+----+----+---+----+---+----+----+----+----+
| C1 | A1 | C2 | A2 | C4 | A4 | 0 | B1 | 0 | B2 | D2 | B4 | D4 |
+----+----+----+----+----+----+---+----+---+----+----+----+----+
\end{verbatim}

  This structure corresponds to the Mode~C altitude reply. This structure is used to encode altitudes above 50187.5 ft.

\end{itemize}

\begin{notebox}{Try it out}
Using \texttt{pyModeS}, we can calculate the altitude code as: 

\begin{verbatim}
import pyModeS as pms

msg = "2000171806A983"
altitude = pms.altcode(msg)  # 36000 (ft)
\end{verbatim}

\end{notebox}


\section{Identity code} \label{sec:id_code}

The 13-bit identity code encodes the 4 octal digit squawk code (from \texttt{0000} to \texttt{7777}). The structure of this field is shown as follows:

\begin{verbatim}
+----+----+----+----+----+----+---+----+----+----+----+----+----+
| C1 | A1 | C2 | A2 | C4 | A4 | X | B1 | D1 | B2 | D2 | B4 | D4 |
+----+----+----+----+----+----+---+----+----+----+----+----+----+
\end{verbatim}

The binary representation of the octal digit is:

\begin{verbatim}
A4 A2 A1 | B4 B2 B1 | C4 C2 C1 | D4 D2 D1
\end{verbatim}


Next, we will use an example to explain the decoding of the identity code. The example message is:

\begin{verbatim}
MSG HEX: 2A00516D492B80
MSG BIN: 00101 01000000000010 1000101101101 010010010010101110000000
         [DF=5]              [identity code]
\end{verbatim}

The binary identity code can be interpreted as follows:

\begin{verbatim}
C1 A1 C2 A2 C4 A4  X  B1 D1 B2 D2 B4 D4
1  0  0  0  1  0   1   1  0  1  1  0  1
\end{verbatim}

Rearranging the bits, we have three groups of binaries:
\begin{verbatim}
A4 A2 A1 | B4 B2 B1 | C4 C2 C1 | D4 D2 D1
0  0  0  | 0  1  1  | 1  0  1  | 1  1  0
\end{verbatim}

Finally, the four octal digit squawk code can be decoded from the binary groups as:
\begin{verbatim}
0 3 5 6
\end{verbatim}

\begin{notebox}{Try it out}
Using \texttt{pyModeS}, we can calculate the squawk code as: 

\begin{verbatim}
import pyModeS as pms

msg = "2A00516D492B80"
altitude = pms.idcode(msg)  # 0356
\end{verbatim}

\end{notebox}