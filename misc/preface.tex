\chapter*{Preface} \label{preface}
I stared my PhD study in 2015 in the aerospace faculty of TU Delft. My research topic was analyzing and modeling aircraft performance using the open aircraft surveillance data. ADS-B data was the primary data source for my research,

Begun with a frustration on the lack of technical public information on ADS-B and Mode-S, I created a live online document to record my understanding and experience of decoding ADS-B data. This was known as the "ADS-B Decoding Guide" project previously. At first it helped me keeping a journal of this new area of research. It become a bit popular after a couple of years. I received questions, feedback, suggestions from the research community all over the world. This greatly helped me fixing and improving the content ADS-B decoding guide over the first two years of my PhD research.

Since the beginning of 2017, the interests of tapping into
Mode-S Enhanced Surveillance (EHS) data brought us a whole new chapter of Mode-S inference and decoding. This also provided more content for the "ADS-B Decoding Guide". With the increasing demand and interests from the readers, I started writing a more comprehensive book - The 1090MHz Riddle - which is focus on the decoding practice of the ADS-B and EHS data.

Alongside the book, a Python library, pyModeS, was also developed, mostly during my spare time, with the contribution of my student Huy Vu, and more other users on the GitHub. The evolution of pyModeS become much faster than the decoding guide, where many new modules in pyModeS are not presented in The previous edition of the book.

In June 2019, I defended my PhD, continued as a post-doc for one and half month. Then, I was appointed as the assistant professor in the aerospace engineering faculty of TU Delft. By now, I had more time to reflect the book and looked at the aircraft surveillance data from a new perspective. Much of the book content can be greatly improved.

That is the reason of this new book edition. It starts from a higher level where the background the aircraft surveillance is discussed. Then I will guide your through the setups and common steps to get a receiver running. The decoding and inference of information are still the core of this book. But more topics beyond the decoding are also discussed.

The book is still published as open access book (with CC BY-NC-SA 4.0 license). It can be easily obtained by searching the title online. The \LaTeX~source of the book is also shared on GitHub, which you can easily adopt to the format that you like.

\begin{flushright}
  Junzi Sun \\
  Delft, the Netherlands \\
  xxxx, 2019
\end{flushright}
