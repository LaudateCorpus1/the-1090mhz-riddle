\chapter*{Preface}

\vspace{-0.3cm}

{
\setstretch{1.03}
This book provides researchers, engineers, and students a practical guide to decoding ADS-B and other types of common Mode~S messages. It consolidates the information from various ICAO documentation and other literature to provide readers easy access to key knowledge of Mode~S and ADS-B and related topics. This book extensively uses examples and sample Python code to explain the decoding process.

Back in 2015, I joined TU Delft to undertake PhD research on analyzing and modeling aircraft performance using open aircraft surveillance data. ADS-B data served as my primary data source. Frustrated with the lack of open literature on ADS-B and Mode~S, I created a live online project to document my experience in decoding ADS-B data, entitled the \emph{ADS-B Decoding Guide}. As the guide grew in popularity, I started receiving questions, feedback, suggestions from the research community all over the world. This input greatly helped me to fix and improve the content of the decoding guide.

At the same time, I also started to incorporate Mode~S Enhanced Surveillance data into my research, which required me to further develop tools for inferring and decoding new types of messages. At the same time, I created more content for the \emph{ADS-B Decoding Guide}. Due to the increasing interest expressed and demand from readers, I started writing a more comprehensive book focused on the decoding practice of the data, which was the starting point for this book.

During these years of work with ADS-B and Mode~S data, I created a Python decoding library, \emph{pyModeS}, which welcomed contributions from GitHub users from all over the world. In this book, some decoding examples are shown using \emph{pyModeS}. However, knowledge of Python is not required to understand the topics covered in this book.

In 2019, I completed my PhD and continued as a faculty member in the aerospace engineering faculty of TU Delft. By then, I had time to reflect on aircraft surveillance data from a new perspective and decided to make a major update of the book's content.

The result is this text, which is both the second edition of \emph{the 1090 Megahertz Riddle} and one of the first books published under the TU Delft's OPEN publishing initiative. The book is published as open access book, under the CC-BY-NC-SA 4.0 license. The \LaTeX~source of the book is also shared on GitHub, where comments and pull requests are greatly appreciated.

\vspace{0.4cm}

\begin{flushright}
  Junzi Sun \\
  Delft, the Netherlands \\
  November, 2020
\end{flushright}
}